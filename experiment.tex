\begin{filecontents*}{refs.bib}
@article{England2013SelfReplication,
  author       = {Jeremy L. England},
  title        = {Statistical physics of self-replication},
  journal      = {The Journal of Chemical Physics},
  year         = {2013},
  volume       = {139},
  number       = {12},
  pages        = {121923},
  doi          = {10.1063/1.4818538},
  url          = {https://doi.org/10.1063/1.4818538}
}

@article{England2015DissipativeAdaptation,
  author       = {Jeremy L. England},
  title        = {Dissipative adaptation in driven self-assembly},
  journal      = {Nature Nanotechnology},
  year         = {2015},
  volume       = {10},
  number       = {11},
  pages        = {919--923},
  doi          = {10.1038/nnano.2015.250},
  url          = {https://doi.org/10.1038/nnano.2015.250}
}

@article{Perunov2016Adaptation,
  author       = {Nikolay Perunov and Robert A. Marsland and Jeremy L. England},
  title        = {Statistical Physics of Adaptation},
  journal      = {Physical Review X},
  year         = {2016},
  volume       = {6},
  number       = {2},
  pages        = {021036},
  doi          = {10.1103/PhysRevX.6.021036},
  url          = {https://doi.org/10.1103/PhysRevX.6.021036}
}

@article{Kolchinsky2024DissipationBound,
  author       = {Artemy Kolchinsky},
  title        = {Thermodynamic dissipation does not bound replicator growth and decay rates},
  journal      = {The Journal of Chemical Physics},
  year         = {2024},
  volume       = {161},
  number       = {12},
  pages        = {124101},
  doi          = {10.1063/5.0213466},
  url          = {https://doi.org/10.1063/5.0213466}
}

@article{teBrinke2018Fibrils,
  author       = {Esra te Brinke and Joost Groen and Andreas Herrmann and Hans A. Heus and Germ{\'a}n Rivas and Evan Spruijt and Wilhelm T. S. Huck},
  title        = {Dissipative adaptation in driven self-assembly leading to self-dividing fibrils},
  journal      = {Nature Nanotechnology},
  year         = {2018},
  volume       = {13},
  number       = {9},
  pages        = {849--855},
  doi          = {10.1038/s41565-018-0192-1},
  url          = {https://doi.org/10.1038/s41565-018-0192-1}
}

@article{Valente2021QuantumDiss,
  author       = {Daniel Valente and Frederico Brito and Thiago Werlang},
  title        = {Quantum dissipative adaptation},
  journal      = {Communications Physics},
  year         = {2021},
  volume       = {4},
  number       = {11},
  pages        = {1--10},
  doi          = {10.1038/s42005-020-00512-0},
  url          = {https://doi.org/10.1038/s42005-020-00512-0}
}

@article{Sartori2015ErrorCorrection,
  author       = {Pablo Sartori and Simone Pigolotti},
  title        = {Thermodynamics of Error Correction},
  journal      = {Physical Review X},
  year         = {2015},
  volume       = {5},
  number       = {4},
  pages        = {041039},
  doi          = {10.1103/PhysRevX.5.041039},
  url          = {https://doi.org/10.1103/PhysRevX.5.041039}
}

@article{BarYam2013PairwiseProfile,
  author       = {Yavni Bar{-}Yam and Dion Harmon and Yaneer Bar{-}Yam},
  title        = {Computationally Tractable Pairwise Complexity Profile},
  journal      = {Complexity},
  year         = {2013},
  volume       = {18},
  number       = {5},
  pages        = {20--27},
  doi          = {10.1002/cplx.21437},
  url          = {https://doi.org/10.1002/cplx.21437}
}

@article{Allen2017MultiscaleInfo,
  author       = {Benjamin Allen and Blake C. Stacey and Yaneer Bar{-}Yam},
  title        = {Multiscale Information Theory and the Marginal Utility of Information},
  journal      = {Entropy},
  year         = {2017},
  volume       = {19},
  number       = {6},
  pages        = {273},
  doi          = {10.3390/e19060273},
  url          = {https://doi.org/10.3390/e19060273}
}

@article{Nag2024PatchyDissipativeAssembly,
  author       = {Suvro Nag and others},
  title        = {Driven Self-Assembly of Patchy Particles Overcoming Kinetic Traps by Dissipative Driving},
  journal      = {Journal of Chemical Theory and Computation},
  year         = {2024},
  volume       = {20},
  number       = {xx},
  pages        = {xxx--xxx},
  note         = {In press},
  doi          = {10.1021/acs.jctc.4c01118},
  url          = {https://doi.org/10.1021/acs.jctc.4c01118}
}

@article{Goldman2025SpreadPatterns,
  author       = {Daniel S. Goldman},
  title        = {Why Some Patterns Spread More Than Others: How Biased Changes and Energy Use Shape Structure in Driven Systems},
  journal      = {Preprint},
  year         = {2025},
  note         = {Manuscript}
}
\end{filecontents*}

\documentclass[11pt]{article}

\usepackage[margin=1in]{geometry}
\usepackage{amsmath,amssymb,amsfonts}
\usepackage{bm}
\usepackage{graphicx}
\usepackage[numbers,sort&compress]{natbib}
\usepackage[colorlinks=true,linkcolor=blue,citecolor=blue,urlcolor=blue]{hyperref}
\usepackage{microtype}

\title{%
Bridging Multiscale Complexity and Replication Thermodynamics:\\
A Concrete Experimental Test of England-Type Self-Replication Bounds%
}

\author{[Author Name]}

\date{\today}

\begin{document}

\maketitle

\begin{abstract}
England's ``statistical physics of self-replication'' derives a lower bound on the heat dissipated during self-replication in terms of a replicator's growth rate, internal entropy, and durability.\citep{England2013SelfReplication} This result has shaped discussions of the thermodynamics of life and dissipative adaptation.\citep{England2015DissipativeAdaptation,Perunov2016Adaptation} At the same time, recent analysis by Kolchinsky argues that the usual interpretation of this bound as a universal relationship between thermodynamic dissipation and per-capita growth and decay rates conflicts with thermodynamic consistency for general replicator models.\citep{Kolchinsky2024DissipationBound} 

Experimental progress on fuel-driven self-assembly and dissipative adaptation already shows that driven systems can reorganize to absorb and dissipate energy effectively,\citep{teBrinke2018Fibrils,Valente2021QuantumDiss,Nag2024PatchyDissipativeAssembly} yet none of these systems has been used as a dedicated test of England's self-replication heat bound. In parallel, multiscale information theory supplies quantitative measures of structure across scales.\citep{BarYam2013PairwiseProfile,Allen2017MultiscaleInfo} Goldman (2025)\citep{Goldman2025SpreadPatterns} develops a bridge between these measures and stochastic thermodynamics: for a Markov process with a designated set of replication-type transitions, the rate at which replication increases multiscale complexity is bounded from below by the entropy production along those transitions. 

Here I build on that bridge to propose a concrete experimental test of England-type bounds in a minimal synthetic replicator. The core system is a DNA templated-ligation replicator in microfluidic droplets, with controlled fuel chemistry and explicit degradation pathways. The bridge framework allows the experiment to use (i) local reaction rates and standard free energies to estimate the entropy production associated with replication, and (ii) statistics of copy-number distributions across droplets to quantify complexity growth. I explain in detail how this design overcomes the main obstacles that blocked direct tests of England's bound, and I show how, in this controlled setting, the bridge experiment amounts to a direct test of the self-replication heat inequality. The paper closes with specific predictions: which trends should appear when chemical driving varies, what signatures would support England-type thermodynamic constraints on replication-driven pattern formation, and what observations would challenge this picture.
\end{abstract}

\section{Introduction}

Self-replication is central to biology and origin-of-life scenarios, and its energetic cost has attracted sustained attention. England's 2013 paper \emph{Statistical physics of self-replication} uses fluctuation theorems to derive a lower bound on the heat released by a self-replicating system coupled to a thermal bath.\citep{England2013SelfReplication} The bound links the minimal heat production rate to the replicator's growth rate, its internal entropy, and its durability. England discusses implications for bacterial cell division and for pre-biotic nucleic acid replicators.\citep{England2013SelfReplication}

That work and related papers on ``dissipative adaptation'' argue that driven many-body systems can evolve toward states that absorb and dissipate work efficiently under a given driving field.\citep{England2015DissipativeAdaptation,Perunov2016Adaptation} This idea has influenced how physicists talk about life's emergence from generic driven matter. At the same time, many presentations of England's theory treat the self-replication bound as a universal relationship between thermodynamic dissipation and per-capita growth and decay rates for any replicator.

Kolchinsky's recent paper challenges that universal reading.\citep{Kolchinsky2024DissipationBound} He shows that a general thermodynamically consistent replicator can grow by consuming fuel and decay into waste products, and that in such cases replication and decay are distinct physical processes with no single, universal relationship between their growth/decay rates and the total entropy production. In that sense, the popular slogan that ``dissipation bounds replicator growth and decay rates'' lacks general validity.

Parallel developments in the thermodynamics of copying and information processing\citep{Sartori2015ErrorCorrection} and in multiscale information theory\citep{BarYam2013PairwiseProfile,Allen2017MultiscaleInfo} supply tools for a more structured approach. Goldman (2025)\citep{Goldman2025SpreadPatterns} introduces a bridge that connects stochastic thermodynamics and multiscale complexity: in a Markov process with a distinguished family of replication-type transitions, the contribution of those transitions to the growth of complexity at scale \(k\) is bounded from below by the entropy production along them.

This bridge suggests a new strategy. Instead of treating England's bound as a sweeping law about growth and decay rates in arbitrary replicators, one can design a controlled synthetic replicator whose reaction network and thermodynamics are known, embed it in a setting where multiscale structure is measurable, and test a direct, quantitative relationship between replication-driven entropy production, multiscale complexity growth, and a self-replication heat inequality in the spirit of England's original derivation.

The aims of this paper are:

\begin{itemize}
  \item to review England's self-replication bound and Kolchinsky's critique in the context of experiments,
  \item to explain how the bridge rewrites the testable content of England's framework in terms of local reaction rates and complexity profiles,
  \item to design a specific droplet-based DNA replicator experiment where all ingredients entering the bound are directly estimated, and
  \item to argue precisely why this constitutes a careful, direct test of a self-replication heat bound in a real physical system, with explicit predictions and possible falsifying outcomes.
\end{itemize}

\section{England's self-replication bound and obstacles to testing}

\subsection{The bound}

England considers a system \(S\) coupled to a heat bath at inverse temperature \(\beta\), evolving under a Markovian dynamics that satisfies local detailed balance.\citep{England2013SelfReplication} One defines two macrostates \(I\) and \(II\), each a metastable region of the system's microstate space, for example corresponding to a situation with one copy of a replicator and a situation with two copies. Over a time interval \(\tau\), the system may undergo a macrotransition \(I\to II\).

Using Crooks' fluctuation theorem and a coarse-graining over microtrajectories that start in \(I\) and end in \(II\), England derives the inequality
\begin{equation}
  \beta \langle Q\rangle_{I\to II} + \Delta S_{\mathrm{int}} 
  \;\ge\; \ln \frac{\pi(I\to II)}{\pi(II\to I)}.
  \label{eq:England_macro}
\end{equation}
Here \(\langle Q\rangle_{I\to II}\) is the average heat released into the bath during such transitions, \(\Delta S_{\mathrm{int}} = S_{\mathrm{int}}(II) - S_{\mathrm{int}}(I)\) is the change in internal entropy of the system between macrostates, and \(\pi(I\to II)\), \(\pi(II\to I)\) are the probabilities that a system initially in \(I\) reaches state \(II\) (or vice versa) within time \(\tau\).\citep{England2013SelfReplication,England2015DissipativeAdaptation}

For a simple birth--death process for copy number \(n\), one can identify macrostates with configurations containing different numbers of copies and relate \(\pi(I\to II)\) and \(\pi(II\to I)\) to growth and decay rates plus a durability factor, arriving at a bound on the minimal heat per replication event in terms of these quantities.\citep{England2013SelfReplication} England uses existing calorimetric and growth-rate data for \emph{E.~coli} to argue that bacterial division operates within a modest factor of this theoretical lower bound.\citep{England2013SelfReplication}

\subsection{Kolchinsky's critique}

Kolchinsky re-examines the inequality \eqref{eq:England_macro} and how it is commonly interpreted.\citep{Kolchinsky2024DissipationBound} The derivation remains valid as a statement about coarse-grained macrostates in a Markov process. The issue concerns the widespread reading that interprets the right-hand side as a function only of per-capita growth and decay rates of a replicator, yielding a universal relationship between dissipation and population dynamics.

Kolchinsky emphasizes that realistic replicators grow by consuming fuel and building structured copies, and decay into waste products rather than into the exact reactants that fuel growth.\citep{Kolchinsky2024DissipationBound} In such cases:

\begin{itemize}
  \item growth and decay occur via distinct reaction channels and reside in different sectors of state space;
  \item a macrostate that represents ``one copy present'' may not have a simple reverse macrostate that represents ``original reactants restored'' with a significant probability in finite time.
\end{itemize}

As a result, the ratio \(\pi(I\to II)/\pi(II\to I)\) that enters \eqref{eq:England_macro} need not be expressible purely in terms of per-capita growth and decay rates defined with respect to a single pair of macrostates. When one attempts to enforce such a form in simple models, the bound often acquires enormous slack.\citep{Kolchinsky2024DissipationBound}

Kolchinsky proves an impossibility theorem: for thermodynamically consistent replicators where decay proceeds to distinct waste states, no universal formula exists that ties both growth and decay rates to total entropy production in the manner often attributed to England's result.\citep{Kolchinsky2024DissipationBound} England's inequality retains value as a coarse-grained fluctuation-theorem statement, yet the usual ``growth/decay vs dissipation'' slogan requires careful qualification.

\subsection{Experimental status}

Several lines of work provide empirical support for aspects of England's broader programme.

First, Sartori and Pigolotti derive universal relations between copy error, entropy production, and dissipated work for biochemical copying and error-correction schemes.\citep{Sartori2015ErrorCorrection} Their framework applies to polymerases and related molecular machines and shows that accuracy, speed, and dissipation obey tight trade-offs independent of microscopic details.

Second, Perunov, Marsland, and England study driven many-body systems and show that under periodic driving, certain macrostates become statistically favored because their trajectories absorb and dissipate work efficiently.\citep{Perunov2016Adaptation} This formalism underpins the ``dissipative adaptation'' concept elaborated in England's 2015 Perspective.\citep{England2015DissipativeAdaptation}

Third, several experiments and simulations display dissipative adaptation in concrete systems. te Brinke and co-workers realize fuel-driven self-assembly of FtsZ inside coacervate droplets, leading to self-dividing fibrils whose morphology and division dynamics depend on chemical driving.\citep{teBrinke2018Fibrils} Valente and colleagues extend the dissipative adaptation scenario to a quantum model and observe analogous behavior.\citep{Valente2021QuantumDiss} Recent simulation studies of patchy-particle systems show driven self-assembly into structures that avoid kinetic traps and efficiently absorb the imposed drive.\citep{Nag2024PatchyDissipativeAssembly}

These systems align with the broad picture of driven matter that self-organizes to dissipate energy. However, none of them measures all ingredients in inequality \eqref{eq:England_macro} in a purpose-designed self-replicator. Existing comparisons of England's bound with biological data largely rely on back-of-the-envelope estimates for bacteria, using aggregate heat production and growth rates.\citep{England2013SelfReplication} A dedicated experiment that creates a minimal replicator with fully characterized reaction network and directly tests a self-replication heat bound remains absent.

\section{A bridge between replication and multiscale complexity}

\subsection{Multiscale complexity profiles}

Multiscale information theory offers a framework to quantify structure at different scales in complex systems.\citep{BarYam2013PairwiseProfile,Allen2017MultiscaleInfo} Consider a set of parts \(A\) (for instance, compartments, lattice sites, or cells), each with a local variable \(X_a\). A microstate is a configuration \(x_A = (x_a)_{a\in A}\). 

Bar-Yam's complexity profile describes how much information applies at each scale of observation.\citep{BarYam2013PairwiseProfile} One construction uses a family of coefficients \(a_B^{(k)}\) to define a scale-\(k\) complexity
\begin{equation}
  C_A(k;t) = \sum_{B\subseteq A} a_B^{(k)} H_t(B),
\end{equation}
where \(H_t(B)\) is the Shannon entropy of \(X_B\) under the distribution \(P_t(X_A)\). The weights \(a_B^{(k)}\) are chosen so that the profile satisfies a sum rule (conservation of total degrees of freedom) and decreases with scale, among other properties.\citep{BarYam2013PairwiseProfile,Allen2017MultiscaleInfo}

In practice, exact evaluation of \(C_A(k;t)\) for large systems is computationally heavy, so pairwise approximations or related constructions are often used.\citep{BarYam2013PairwiseProfile,Allen2017MultiscaleInfo} These still capture essential features: \(C_A(1;t)\) reflects the diversity of individual part states, while \(C_A(k;t)\) for \(k>1\) reflects correlations and higher-order structure.

\subsection{Stochastic thermodynamics and replication edges}

Goldman (2025) considers a continuous-time Markov process on a discrete microstate space \(\Omega\), with rate matrix \(W_{\omega\omega'}\) and state probabilities \(P_t(\omega)\).\citep{Goldman2025SpreadPatterns} Each microstate determines a configuration \(X_A(\omega)\) of part variables.

Under local detailed balance, each transition \(\omega\to\omega'\) carries a log-rate asymmetry
\begin{equation}
  \sigma_{\omega\omega'} = \ln\frac{W_{\omega\omega'}}{W_{\omega'\omega}},
\end{equation}
which equals the entropy flow into the environment (in units of \(k_{\mathrm{B}}\)) plus any contributions from non-thermal reservoirs. Summing over all transitions yields the total entropy production rate
\begin{equation}
  \Sigma_{\mathrm{tot}}(t) = 
  \frac{1}{2}\sum_{\omega,\omega'} 
  J_{\omega\omega'}(t)\,\sigma_{\omega\omega'}, 
\end{equation}
where \(J_{\omega\omega'}(t) = P_t(\omega)W_{\omega\omega'} - P_t(\omega')W_{\omega'\omega}\) is the probability current.

Goldman introduces a distinguished set \(R\) of \emph{replication-type} transitions: those that increase the abundance or presence of a particular pattern (for example, creation of a template molecule via templated ligation). The entropy production rate along these edges is
\begin{equation}
  \Sigma_R(t) = 
  \sum_{(\omega,\omega')\in R} 
  P_t(\omega)W_{\omega\omega'}\,\sigma_{\omega\omega'}.
\end{equation}

\subsection{Information flux and the bridge inequality}

To connect dynamics to multiscale complexity, Goldman defines an informational force
\begin{equation}
  K_k(\omega,t) = 
  \sum_{B\subseteq A} a_B^{(k)} 
    \log P_t(X_B = X_B(\omega)).
\end{equation}
This quantity measures how ``typical'' the configuration of each subset \(B\) is within the current ensemble; it plays the role of a generalized potential.

Goldman shows that the time derivative of the multiscale complexity can be written as an information flux functional:
\begin{equation}
  \frac{d}{dt} C_A(k;t)
  = \sum_{\omega,\omega'} 
      P_t(\omega)W_{\omega\omega'}
     \bigl[ K_k(\omega',t) - K_k(\omega,t)\bigr].
  \label{eq:info_flux}
\end{equation}
The contribution of replication edges to complexity growth is
\begin{equation}
  \dot C_A^{(R)}(k;t)
  = \sum_{(\omega,\omega')\in R} 
      P_t(\omega)W_{\omega\omega'}
     \bigl[ K_k(\omega',t) - K_k(\omega,t)\bigr].
\end{equation}

Goldman then proves that for each scale \(k\) and time \(t\),
\begin{equation}
  \dot C_A^{(R)}(k;t) 
  \;\ge\; r_{\min}(k,t)\,\Sigma_R(t),
  \label{eq:bridge_ineq}
\end{equation}
where \(r_{\min}(k,t)\) is the minimal ratio of informational force to thermodynamic force over replication edges at that time.\citep{Goldman2025SpreadPatterns} This inequality states that whenever replication-type transitions increase multiscale complexity at scale \(k\), they consume entropy at a rate at least \(\Sigma_R(t)/r_{\min}(k,t)\).

In the context of England's work, this bridge has two key consequences:

\begin{enumerate}
  \item It isolates the entropy production associated specifically with replication-like transitions, in contrast to the total entropy production of the system.
  \item It expresses the structural outcome of replication in terms of multiscale complexity growth, which can be estimated from ensemble statistics across many realizations (e.g., compartments) instead of requiring direct calorimetry.
\end{enumerate}

The experimental proposal in the next sections exploits these features.

\section{A droplet-based DNA templated-ligation replicator}

\subsection{Design principles}

To test a self-replication heat bound in a way that aligns with England's derivation and Kolchinsky's critique, the experimental system should satisfy several criteria:

\begin{enumerate}
  \item \textbf{Minimal, explicit reaction network.} The replicator should arise from a small set of well-characterized reactions, with known stoichiometry and standard free energies.
  \item \textbf{Controlled reverse pathway.} For at least one replication channel, there should exist a well-defined reverse reaction that restores the reactants, so that the ratio of forward and backward transition probabilities can be related cleanly to thermodynamic parameters.
  \item \textbf{Separated decay to waste.} Degradation into waste should be easily distinguished in the reaction network, so that replication and decay edges can be separated, addressing the concerns raised by Kolchinsky.\citep{Kolchinsky2024DissipationBound}
  \item \textbf{Compartmentalization.} Replication should occur in many small, quasi-independent compartments (droplets), each serving as a ``part'' in the multiscale complexity analysis.
  \item \textbf{Accessible observables.} It should be possible to measure copy-number statistics across compartments, reaction rates, and fuel consumption with established techniques.
\end{enumerate}

DNA templated ligation in microdroplets provides a promising platform that meets these criteria.

\subsection{Chemistry of the templated-ligation replicator}

Consider a DNA sequence \(T\) that can act as a template for its own formation by ligation of two complementary half-strands \(h_1\) and \(h_2\). The basic reactions in each droplet are:

\begin{align}
  h_1 + h_2 + T + \text{fuel} 
  &\xrightleftharpoons[k_{-}]{k_{+}} 
    T + T + \text{waste} 
    \label{eq:templated_ligation}
  \\
  T &\xrightarrow{k_{\mathrm{deg}}} W.
    \label{eq:degradation}
\end{align}

Reaction \eqref{eq:templated_ligation} represents templated ligation: two half-strands hybridize to an existing template and ligate to form a new copy, consuming a fuel molecule (for instance ATP) and generating a waste product (ADP + Pi). The reverse reaction corresponds to de-ligation combined with fuel regeneration; its rate can be strongly suppressed by the chemical conditions, yet it remains well defined in principle. Reaction \eqref{eq:degradation} represents degradation of \(T\) into inert waste strands \(W\) by nuclease activity or strand displacement.

The free energy change \(\Delta G_{\mathrm{rep}}\) for reaction \eqref{eq:templated_ligation} can be estimated from:

\begin{itemize}
  \item the standard free energy of ligation,
  \item base-pairing and stacking contributions for the duplex,
  \item the chemical potential drop associated with fuel consumption.
\end{itemize}

Under local detailed balance, the log-rate asymmetry for the ligation step satisfies
\begin{equation}
  \sigma_{\mathrm{rep}} 
  = \ln\frac{k_{+}}{k_{-}} 
  = -\beta \Delta G_{\mathrm{rep}}.
\end{equation}

By adjusting fuel concentration and buffer conditions, one can control \(\Delta G_{\mathrm{rep}}\) and thereby tune the driving strength.

Eq.~\eqref{eq:templated_ligation} implements a physically reversible replication channel in the sense required by England's derivation. Eq.~\eqref{eq:degradation} provides a separate decay channel into waste, which can be treated distinctly.

\subsection{Microfluidic droplets as parts}

Droplet microfluidics allows generation of large numbers of monodisperse water-in-oil droplets that serve as independent reaction compartments.\citep[for general techniques, see e.g.][]{teBrinke2018Fibrils} Each droplet contains the same buffer, fuel molecules, \(h_1\), \(h_2\), ligase, nuclease, and a very low initial concentration of \(T\). Conditions are chosen so that most droplets start with zero or one template molecule.

Each droplet \(a\in A\) has a state variable \(X_a(t)\) that encodes its template content. Several choices are suitable:

\begin{itemize}
  \item a discrete copy number \(n_a(t)\in\{0,1,2,\dots\}\),
  \item a binary variable \(X_a(t)\in\{0,1\}\) indicating whether \(n_a(t)\) exceeds a threshold,
  \item a small set of categories for ranges of \(n_a(t)\).
\end{itemize}

Fluorescent probes (e.g.\ sequence-specific dyes or molecular beacons) provide a readout of \(n_a(t)\) for each droplet at selected time points. By imaging thousands of droplets, one obtains empirical estimates of the joint distribution \(P_t(X_A)\) and thus the multiscale complexity profile \(C_A(k;t)\) (or a suitable approximation).\citep{BarYam2013PairwiseProfile,Allen2017MultiscaleInfo}

\subsection{Kinetic and thermodynamic characterization}

In parallel with the droplet ensemble experiments, bulk or single-droplet kinetic measurements determine the forward and reverse rates \(k_{+}\) and \(k_{-}\) for templated ligation, and the degradation rate \(k_{\mathrm{deg}}\). Combined with standard thermodynamic data for the underlying chemical reactions, this yields estimates for \(\Delta G_{\mathrm{rep}}\) and for the log-rate asymmetry \(\sigma_{\mathrm{rep}}\).

By monitoring fuel and waste concentrations over time in a companion bulk reaction, one can verify that the estimated per-ligation heat release aligns with calorimetric data and with measured reaction energetics, providing a cross-check on the thermodynamic parameters.

Within the Markov-process description, each ligation event in droplet \(a\) corresponds to a replication edge \(\omega\to\omega'\in R\) that increments \(n_a\) by one. The contribution of this event to entropy production is approximately \(\sigma_{\mathrm{rep}}\). Summed over all droplets and events, the replication-edge entropy production rate satisfies
\begin{equation}
  \Sigma_R(t) \approx J_{\mathrm{rep}}(t)\,\sigma_{\mathrm{rep}},
\end{equation}
where \(J_{\mathrm{rep}}(t)\) is the net replication flux (number of ligation events per unit time).

\section{How the bridge experiment tests a self-replication heat bound}

This section argues in a step-by-step way how the proposed experiment implements a careful, direct test of a self-replication heat bound, and how the bridge framework is central to that claim.

\subsection{From England's macro-inequality to a local, replication-focused version}

Inequality \eqref{eq:England_macro} arises from coarse-graining over microtrajectories that start in macrostate \(I\) and end in macrostate \(II\) in time \(\tau\).\citep{England2013SelfReplication} The ratio \(\pi(I\to II)/\pi(II\to I)\) is a log-likelihood ratio between forward and reverse macrotransitions.

In the templated-ligation replicator, a natural choice of macrostates for a single droplet is:

\begin{itemize}
  \item \(I\): the droplet contains exactly one template molecule \(T\) and a fixed set of half-strands and fuel molecules, with no waste.
  \item \(II\): the droplet contains exactly two template molecules and the appropriate amount of waste products from one ligation, with the same number of free half-strands as in \(I\).
\end{itemize}

The transition \(I\to II\) in time \(\tau\) corresponds to a ligation event mediated by the existing template. The reverse macrotransition \(II\to I\) corresponds to de-ligation plus fuel regeneration (the reverse of reaction \eqref{eq:templated_ligation}), with an appropriate microtrajectory that restores the original chemical composition.

In this minimal system:

\begin{itemize}
  \item the forward and reverse macrotransitions are dominated by the single chemically reversible reaction channel with free energy change \(\Delta G_{\mathrm{rep}}\);
  \item decay into waste via reaction \eqref{eq:degradation} is an additional escape channel that can be modeled explicitly and kept small on the timescale of interest.
\end{itemize}

The ratio of macrotransition probabilities \(\pi(I\to II)/\pi(II\to I)\) over a suitably chosen time window can then be expressed in terms of the microscopic rates \(k_{+}\) and \(k_{-}\) and any additional fast equilibration within each macrostate. Under the conditions described above, the log-ratio approximates \(\sigma_{\mathrm{rep}}\), up to small corrections.

For this engineered replicator, England's inequality \eqref{eq:England_macro} therefore reduces to
\begin{equation}
  \beta\langle Q\rangle_{I\to II} + \Delta S_{\mathrm{int}}
  \;\gtrsim\; \sigma_{\mathrm{rep}},
  \label{eq:local_bound_simple}
\end{equation}
where \(\langle Q\rangle_{I\to II}\) is the mean heat released in a single templated-ligation event and \(\Delta S_{\mathrm{int}}\) is the change in internal entropy between macrostate \(I\) and macrostate \(II\). Both sides of \eqref{eq:local_bound_simple} involve quantities accessible in the experiment:

\begin{itemize}
  \item \(\langle Q\rangle_{I\to II}\) follows from \(\Delta G_{\mathrm{rep}}\) and the measured entropy flow into the environment per replication event.
  \item \(\Delta S_{\mathrm{int}}\) reflects the entropy change of the droplet contents; for a minimal template replicator with fixed composition aside from template number and waste, this is small and can be bounded from thermodynamic data and simple combinatorial considerations.
  \item \(\sigma_{\mathrm{rep}} = \ln(k_{+}/k_{-}) = -\beta \Delta G_{\mathrm{rep}}\) comes from kinetic measurements and standard free energies.
\end{itemize}

In this sense, the experiment directly compares the left-hand side of a self-replication heat inequality derived in England's style with the right-hand side obtained from local rate asymmetries.

\subsection{Role of the bridge: from macrotransition ratio to replication-edge entropy production}

The bridge contributes two crucial pieces to this argument.

First, by isolating the family of replication-type transitions \(R\) and defining the replication-edge entropy production rate \(\Sigma_R(t)\), the bridge expresses the relevant log-likelihood ratios in terms of local rate asymmetries and fluxes:\citep{Goldman2025SpreadPatterns}
\begin{equation}
  \Sigma_R(t) = \sum_{(\omega,\omega')\in R} P_t(\omega)W_{\omega\omega'}\,\sigma_{\omega\omega'}.
\end{equation}
For a system where replication between \(I\) and \(II\) proceeds mainly via a single reversible reaction channel \eqref{eq:templated_ligation} with log-rate asymmetry \(\sigma_{\mathrm{rep}}\), the integrated replication-edge entropy production over a time interval \([0,\tau]\) satisfies
\begin{equation}
  \int_0^\tau \Sigma_R(t)\,dt 
  \approx N_{\mathrm{rep}}(\tau)\,\sigma_{\mathrm{rep}},
\end{equation}
where \(N_{\mathrm{rep}}(\tau)\) is the expected number of replication events in that interval, summed over droplets. For a single droplet, the corresponding quantity equals the log-likelihood ratio between forward and backward replication paths, in close analogy with the macrotransition ratio in England's derivation.

Second, the bridge inequality \eqref{eq:bridge_ineq} shows that this entropy production is exactly the budget available to change multiscale complexity via replication edges:
\begin{equation}
  \dot C_A^{(R)}(k;t) \ge r_{\min}(k,t)\,\Sigma_R(t).
\end{equation}
Thus the same replication-edge entropy production that appears in a local version of England's heat bound simultaneously constrains the growth of multiscale structure across the droplet array.

Combining these points yields a three-way linkage in the experimental system:

\begin{enumerate}
  \item the per-event self-replication heat bound in \eqref{eq:local_bound_simple},
  \item the replication-edge entropy production \(\Sigma_R(t)\) and its integral over time, obtained from kinetics and thermodynamics, and
  \item the replication-driven growth of multiscale complexity \(\dot C_A^{(R)}(k;t)\), obtained from copy-number statistics across droplets.
\end{enumerate}

The bridge defines \(\Sigma_R(t)\) in a way that matches both the local detailed-balance structure of England's derivation and the complexity profile. This alignment is what turns the experiment into a direct test of a self-replication heat bound rather than a looser, purely phenomenological study of pattern formation.

\subsection{Addressing Kolchinsky's objections in the experimental design}

Kolchinsky's impossibility result applies when one tries to relate total entropy production, per-capita growth, and per-capita decay rates of a general replicator that decays into waste.\citep{Kolchinsky2024DissipationBound} The engineered DNA system deliberately avoids this trap in two ways:

\begin{itemize}
  \item The core replication channel \eqref{eq:templated_ligation} has a well-defined reverse reaction that restores the original reactants (half-strands plus fuel). This gives a genuine forward and reverse macrotransition between \(I\) and \(II\) in the sense of England's fluctuation-theorem derivation, without invoking decay into waste.
  \item Decay into waste via \eqref{eq:degradation} is present yet modeled separately. It provides an additional sink that can be characterized and either kept rare on the timescale of the self-replication test or included explicitly in the Markov description. When one focuses on the ligation channel as the object of England's inequality, decay events lie outside the chosen macrotransition pair.
\end{itemize}

The bridge framework, with its explicit separation of replication edges \(R\) and other transitions, mirrors this design. By restricting attention to entropy production along \(R\), the experiment tests a self-replication heat bound for a physically realized replication channel that exhibits a clear reverse process. This setup sidesteps the generic waste-decay motif that Kolchinsky uses in his impossibility argument, while still aligning with his call for thermodynamically consistent modeling.

\subsection{Why this qualifies as a careful, direct test}

The experiment qualifies as a careful, direct test of a self-replication heat bound in several senses:

\begin{itemize}
  \item \textbf{All bound ingredients are measured or tightly estimated.} The mean heat per replication event \(\langle Q\rangle_{I\to II}\) follows from standard free energies and fuel turnover; \(\Delta S_{\mathrm{int}}\) can be computed for the minimal droplet composition; the log-rate asymmetry \(\sigma_{\mathrm{rep}}\) comes from kinetic measurements.
  \item \textbf{Thermodynamic environment is controlled.} Droplets are in a well-defined thermal bath at fixed temperature, with controlled fuel and substrate concentrations. Side reactions can be minimized and characterized.
  \item \textbf{The Markov and local detailed-balance assumptions are justified.} Reaction networks with DNA ligation and hybridization have been modeled successfully within stochastic thermodynamics; the local detailed-balance relation between rate asymmetries and free energy changes is standard in that context.
  \item \textbf{Macrostate definitions match the chemistry.} Macrostates \(I\) and \(II\) differ only by template copy number and the associated fuel/waste molecules. The reverse macrotransition is implemented by the actual chemical reverse reaction for templated ligation, rather than by an abstract decay process to a different sector of state space.
  \item \textbf{The bridge supplies a consistency check through complexity.} Beyond comparing \(\beta \langle Q\rangle_{I\to II} + \Delta S_{\mathrm{int}}\) with \(\sigma_{\mathrm{rep}}\) directly, the experiment can also verify that the replication-edge entropy production extracted from kinetics consistently accounts for the observed increase in multiscale complexity, as inequality \eqref{eq:bridge_ineq} demands.
\end{itemize}

A successful result would therefore amount to a direct test of an England-style self-replication heat inequality in a real physical system, with clear thermodynamic bookkeeping and structural observables.

\section{Predictions and signatures in the droplet experiment}

This section summarizes the main predictions shaped by the bridge model and highlights what one should and should not see when varying the chemical driving.

\subsection{Early-time regime and small-scale complexity}

Focus first on early times, when each droplet contains zero, one, or a few templates and degradation remains rare. In this regime, droplet dynamics approximate a branching process driven by reaction \eqref{eq:templated_ligation}. Let \(C_A(1;t)\) denote the scale-1 complexity, which captures the diversity of droplet states.

\paragraph{Prediction 1: complexity growth tracks replication-edge entropy production.}

At early times and moderate driving, the rate of increase of \(C_A(1;t)\) should scale with \(\Sigma_R(t)\). Under reasonable approximations,
\begin{equation}
  \dot C_A(1;t) \approx \alpha(t)\,J_{\mathrm{rep}}(t),
\end{equation}
for some positive factor \(\alpha(t)\) that encodes how each replication event changes the droplet-state distribution. Since \(\Sigma_R(t) \approx J_{\mathrm{rep}}(t)\,\sigma_{\mathrm{rep}}\), the bridge inequality yields
\begin{equation}
  \dot C_A(1;t) 
  \;\gtrsim\; r_{\min}(1,t)\,\Sigma_R(t),
\end{equation}
and one expects an approximately linear relationship between \(\dot C_A(1;t)\) and \(\Sigma_R(t)\) across a family of driving conditions at fixed \(t\). When driving is extremely weak, both quantities should approach zero together.

In experiments, this means that when the free energy drop per replication \(|\Delta G_{\mathrm{rep}}|\) and the replication flux \(J_{\mathrm{rep}}(t)\) increase, the early-time growth rate of diversity across droplets should increase in step. Sustained growth of \(C_A(1;t)\) at vanishingly small \(\Sigma_R(t)\) would contradict the bridge picture.

\subsection{Integrated complexity gain and entropy budget}

Over a finite time window \([t_0,t_1]\), the total change in complexity at scale \(1\) due to replication edges is
\begin{equation}
  \Delta C_A^{(R)}(1)
  = \int_{t_0}^{t_1} \dot C_A^{(R)}(1;t)\,dt.
\end{equation}
Integrating \eqref{eq:bridge_ineq} gives
\begin{equation}
  \Delta C_A^{(R)}(1)
  \;\ge\; \int_{t_0}^{t_1} r_{\min}(1,t)\,\Sigma_R(t)\,dt.
\end{equation}
For a set of experiments with different driving strengths, one can estimate for each condition an effective slope
\begin{equation}
  r_{\mathrm{eff}}(1) \approx
  \frac{\Delta C_A^{(R)}(1)}{\int_{t_0}^{t_1} \Sigma_R(t)\,dt}.
\end{equation}

\paragraph{Prediction 2: a lower envelope in the complexity--entropy plane.}

When plotting \(\Delta C_A^{(R)}(1)\) against \(\int_{t_0}^{t_1} \Sigma_R(t)\,dt\) across conditions, points should lie above a roughly linear lower envelope with nonzero slope. This envelope estimates \(\min r_{\mathrm{eff}}(1)\) across conditions, which in turn bounds the minimal conversion of replication-edge entropy production into small-scale complexity. Parameter regimes that repeatedly fall far below this envelope, without clear changes in other experimental features, would signal a breakdown of the assumptions that underlie the bridge (for example, misidentification of replication transitions or an incomplete account of thermodynamic driving).

\subsection{Effects of increased driving and saturation}

As the driving strength \(|\Delta G_{\mathrm{rep}}|\) increases further, several effects arise:

\begin{itemize}
  \item The replication rate \(k_{+}\) and thus \(J_{\mathrm{rep}}(t)\) increase until limited by substrate depletion or crowding.
  \item Degradation and reverse reactions may remain comparatively slow, so entropy production per event grows roughly in proportion to \(|\Delta G_{\mathrm{rep}}|\).
  \item The droplet population eventually becomes dominated by template-rich states, which can reduce diversity at scale~1.
\end{itemize}

\paragraph{Prediction 3: diminishing returns and possible decline of \(C_A(1)\) at high driving.}

In the complexity--entropy plane, this regime should appear as a bending of the curve: \(\int \Sigma_R dt\) continues to grow with increased driving, yet \(\Delta C_A^{(R)}(1)\) saturates or even declines as the droplet ensemble becomes homogeneous in template content. The ratio \(r_{\mathrm{eff}}(1)\) decreases in this regime, illustrating that additional replication-driven entropy production no longer yields proportional gains in small-scale complexity.

This behavior aligns with intuition: extremely strong driving can push the system into a highly ordered yet low-diversity state, such as every droplet saturated with templates.

\subsection{Higher-scale structure}

If droplets experience spatial variations in initial conditions, driving strength, or external fields, replication can generate patterns across the droplet array that involve correlations among neighboring compartments. Examples include gradients of template-rich droplets or clusters embedded in a template-poor background.

\paragraph{Prediction 4: higher-scale complexity correlates with increased entropy budget.}

For scales \(k>1\), the bridge inequality
\begin{equation}
  \dot C_A^{(R)}(k;t) \ge r_{\min}(k,t)\,\Sigma_R(t)
\end{equation}
implies that sustained increases in higher-scale complexity require at least proportional growth in replication-edge entropy production. Experiments that engineer spatial heterogeneity and then measure both \(C_A(k;t)\) and \(\Sigma_R(t)\) can test whether the emergence of multi-droplet patterns comes with an appropriate thermodynamic cost.

\subsection{What should not happen if the bridge and bound hold}

Several qualitative behaviors would conflict with the combo of England-style self-replication inequalities and the bridge:

\begin{enumerate}
  \item Persistent growth of \(C_A(1;t)\) in regimes where kinetic and thermodynamic measurements indicate negligible \(\Sigma_R(t)\), after accounting for all plausible sources of thermodynamic driving.
  \item The ability to tune control parameters such that \(\Delta C_A^{(R)}(1)\) stays large while \(\int \Sigma_R dt\) shrinks arbitrarily, without clear changes in reaction network structure or hidden energy flows.
  \item Emergence of higher-scale structure (\(k>1\)) that depends sensitively on replication events, yet fails to show any trend with \(\Sigma_R\) across driving conditions.
\end{enumerate}

These outcomes would suggest either missing thermodynamic channels, invalid model assumptions (for example, a breakdown of the Markov approximation), or a deeper issue with the conceptual link between replication, entropy production, and complexity growth.

\section{Discussion and outlook}

The proposed experiment uses the bridge to redesign how England-type self-replication bounds are tested. Instead of seeking a universal relationship between total entropy production and growth/decay rates for arbitrary replicators, the experiment concentrates on a single, physically reversible replication channel in a minimal DNA system, with explicit separation of decay to waste.\citep{Kolchinsky2024DissipationBound} 

This focus yields several advantages:

\begin{itemize}
  \item The self-replication heat inequality \eqref{eq:local_bound_simple} becomes directly testable because both sides involve quantities accessible through standard biochemical and thermodynamic measurements.
  \item The bridge's replication-edge entropy production \(\Sigma_R(t)\) links the same thermodynamic budget to multiscale complexity growth across droplets, providing an internal consistency check that goes beyond calorimetry.
  \item The multiscale complexity profile connects this synthetic system to broader ideas about scale-dependent structure and requisite variety.\citep{BarYam2013PairwiseProfile,Allen2017MultiscaleInfo,Goldman2025SpreadPatterns}
\end{itemize}

Success would deliver a concrete demonstration that replication-driven entropy production constrains the rate and manner in which patterns spread in a driven chemical system, in a way that echoes England's Markov-thermodynamic logic while respecting Kolchinsky's thermodynamic consistency requirements. It would also create a template for more complex systems, such as RNA replicators, coacervate protocells, or artificial cells with coupled replication and division.\citep{teBrinke2018Fibrils,Valente2021QuantumDiss,Nag2024PatchyDissipativeAssembly}

Conversely, if careful experiments reveal systematic departures from the predicted relationships---for example, large complexity gains with tiny replication-edge entropy budgets even after exhaustive thermodynamic accounting---the outcome would clarify where current theoretical frameworks need revision. Either way, the bridge turns a previously diffuse debate about dissipation and replication into a program with specific experimental milestones.

\section*{Acknowledgments}

[Add any acknowledgments here.]

\bibliographystyle{plainnat}
\bibliography{refs}

\end{document}
